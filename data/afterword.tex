\section*{苏靖云}
由于这次调研支队人数较少,支队成员们都需要承担很多工作,好在大家也都很用心,让这次实践顺利完成,在整个实践过程中,也收获了很多。

前期准备是我觉得最困难的时候,联系调研公司的时间稍微晚了点,本以为应该较容易找到,却多次失败:发邮件没有回复,打电话会被公司前台以各种理由拒绝,最后自己只联系上了一家公司,并借助系办帮忙联系上了另外两家,才勉强可以出发,这也使得支队在申请各种项目上落后。同时由于刚开始联系的公司最终没有回复,让支队的第一次会议基本作废,也让队员做了许多无用功,这也是前期支队组织上存在的问题,好在经过这次实践,以后应该会改善不少。

在调研的过程中,我们碰到了三家较为典型的企业,第一家福元药业也并非所想的那么顺利,与预想差距较大,这也正好能丰富我们的见识,并发现自己所作的准备的不足。之后的企业便是一家比一家好,对于访谈问题的提出方式、顺序及调研稿的整理也越来越有经验。同时有了第一家企业调研结束后的总结,我们对于提问流程进行了改善,在后两家企业的访谈中也成功避免了一些尴尬的局面,队员间的配合较为默契。不过在后来的访谈中,由于公司采取了不同的模式,也打乱了我们提问节奏,导致部分问题被一带而过,这也对后期调研报告的书写造成了一定的阻碍。

本以为前期调研已经做得足够充分,但在写调研报告时才发现,之前的调研只是增加了对行业的了解,但所查到的数据大多已经过时,因此在后期不得不重新查找最新的数据。除此之外,调研成功的总结还算比较顺利,只是在统一大家报告及ppt制作上消耗了不少时间,队员们也比较积极,很快地完成了写报告的工作。

这次调研的福元药业,先声药业,合全药业都具有一定的代表性,也让我通过此次调研了解了制药企业本身的运作方式以及国内制药行业的现状。尽管国家已经大力提倡创新,仍然存在福元药业的封闭型企业,他们通过大量仿制获取利益,由于自身研发人员学历等,研发设施及技术等多方面限制,难以做出较好的仿制药或首仿药物,只能通过数量来弥补质量上的不足,他们也必然会对同一类型产品进行大量仿制,这只会使得沟内药物越来越杂,以及类似企业间的恶性竞争。不过随着一致性评价及“4+7”的推行,这种多种药物针对同一种疾病的现象将会得到缓解,而类似福元药业的企业也会受到较大冲击,大量小型企业将会被淘汰。先声药业则是进行创新后不断尝试新的方法推动企业进步,如对小型企业进行投资,像新的医药领域进行拓展等,这也是其他企业可以参考的地方,也只有这样,我们才能逐渐拉近与国外医药行业的距离,生物制药如今国内已经逐渐发展,但精准医疗领域似乎仍然落后,需要继续发展。合全药业作为较为特殊的类型,在国内已经处于顶尖地位,因此只需要继续按照原本发展模式,跟进国际上先进技术,同时继续扩展业务面,促进公司发展。因此未来一段时间内,将会有大量小型企业遭到淘汰,小部分被迫转型或被合并,逐渐发展壮大,这也能提高国内医药行业产业集中度,增加上市药物质量。


\section*{戚鉴清}

\paragraph{关于前期调研}
前期调研中查阅文献和专利的工作对于我来说倒没有那么难,但是联系公司的时候就会感到比较烦躁啦。首先我们很难找到对的联系方式:官网上的电话往往是怕事的前台,根本不会理会我们的请求;更多的时候是机器人客服,直接要求我们输入分机号……就算在少数情况下死皮赖脸地问道了人力资源部或者其他部门的电话,也会被这几个部门来回踢皮球,最后调研的事情只得不了了之。还好运气比较好,找到了像上海合全药业有限公司这样靠谱的企业,我们的调研内容才能变得如此充实。

我感觉我们支队在前期调研的时候视野还是不够开阔,带着太强的学生心态,完全不了解企业实际是怎样的。虽然这完全不能怪罪到任何一个支队员身上,但是我们对于企业内部运作原理的无知,多多少少还是带来了挺多不便的。

\paragraph{关于调研过程中}
我个人对于调研过程整体还是比较满意的,我们在行程中没有遇到什么麻烦,没有出现迟到和安全问题。

不过在谈到具体的访谈环节的时候,我感觉我们支队在前期的演练还是不够。我们在问问题的时候卡壳和不连续的现象还是太多了,同时对于企业给出的回答缺少一定的预期,有时出现接不上话的情况。以后如果还做调研类实践的话,需要多放一些时间到问题的优化上。

\paragraph{关于调研结果}
不得不说,我们支队的成员都是非常给力效率非常高的。大家对于调研这件事都非常上心,同时每个人也有独到的思考和分析,调研报告的撰写还是非常顺利的。在行程中写推送的时候大家也非常积极,花费了相当大的精力来完成这件事情,让我感受到一群人团结一致时的效率能有多高。


\section*{石强}
短短的一周时间,三地奔波,在三家企业之间调研访谈。此间,尽管只是小小的缩影,却依旧感受到了医药企业当下的境况。

整个实践过程中,从福元再到先声与合全,随着公司规模的提升,随着公司发展方向的不同变化,以及其他多样化的原因,对于制药行业的认知也在一步步深入着。因此,访谈期间,时间推移之下,所体会到的是在相同的行业背景下,小公司受制于人才、设备等自身因素,而处在如今快速波动而又日益分工化的行业状况下,它们所面临的,是对于自我定位的更加细致化的明确,否则,要是像以往那样想要覆盖整个领域,那带来的代价势必是降低自己的产品质量,如此下去,就将会是一次恶性循环。长此以往,它们的生存必然会处于风雨飘摇之中。因此,对于小公司来说,夹缝之中的生存实为不易,找寻出路变为了极为残酷的问题。

而对于先声和合全这样的大企业来说,由于各自的定位不同——先声更多的在提高制药源头的创新能力,合全则是充分利用了分工的大背景选择了承担外包工作执行研发生产方面。在有所选择之后,对于公司本身来说,可以更加集中性地投入相应的资源,如完成团队的组建。那么,对于大公司来说,更多的就应当考虑到自己之后的发展方向,做好当下背景与未来的衔接。同时,作为中国的一线企业,它们也都承担有相应的社会责任,并且帮助一定的小型企业整合资源完成企业的互助,这一点也颇为不错。

此外,在参观的过程中,可以明显地感受到,公司的实力直接影响到了科研开发人员的工作环境,同时,也会对于之后的研发带来比较大的影响。因此,在全新的社会与行业背景下,我觉得,对于个人来说,更为重要的是去发展自我,由于专业性增强,那么对于人才的需求必然更加强烈,这一点尤为突出。

当下中国的制药企业正面临着多样化的境况,究竟是增强自己的多方面能力,还是去彻底转型成为分工,这都会是选择。但可以明确的一点是,无论如何,中国的制药企业正在走入新的阶段,这既是生机的源泉,同时也会是风险的缝隙,这一切,都将最终取决于创新能力的强弱了。


\section*{皇甫硕龙}

通过这次调研实践,我主要深入认识和了解了制药企业的企业决策和制药行业的行业分工。通常来讲,制药行业的成本高,研发周期长,从投入到成本回收历时长,因此许多中小型企业基于自身研发实力选择放弃进行研发而转向投入仿制药。虽然从企业的角度看,这样的举措可能会在短期内为企业带来较为显著的收益,能达到“赚快钱”的目标。长期来看,仿制药的泛滥无论是对制药行业还是对消费者都有害处。低端仿制药的大量出现会抑制整个制药行业的创新能力,从而导致自主产权创新药的减少,而国外企业针对中国市场可以利用价格歧视政策扩大利润,整体来看结果对消费者不利。另一方面,药品是一种生活必需品,关系到人们的健康与寿命,过高或不稳定的药品价格会造成社会不稳定因素,也会促使药品市场中假冒药品的泛滥。

和中小型企业的决策不同的是,药明康德集团化和合全药业研发外包的措施可以有效激励药物研发的创新。借助这样的生产模式,药物不同阶段的研发可以分别由不同的企业完成。相比于传统制药企业不同研发部门中可能存在的组织问题,在组织上,这些企业会针对研发环节的特点来组织人员以最大化研发效率,从而使得研发成本降低。在这次调研中,我认识到市场上企业间的合作、供应链中的分工往往可以降低整体的生产成本,缩短药物的研发周期。在整个行业中,大型制药企业应当为中小型制药企业的产业化提供平台。如先声药业的创新药项目百家汇为创新企业提供了一个良好的发展环境。利用开放的创新模式促进药物的研发。

宏观政策的调控也会对制药企业的决策产生影响,近来国家出台一系列缩减药物审批流程和简化药物临床试验流程的政策,在一定程度上起到了降低研发成本的作用,有效地鼓励制药企业进行研发创新。


\section*{白雪杨}
这次实践前我对整个制药行业的情况知之甚少,从实践的前期调研到实地探访,我不仅了解了制药行业的生产流程等基本情况,也在与从业者的亲身交流中看到了这个行业在国内发展的现状与前景,而且也算是了解到了化学系同学毕业后有可能的一种与学校实验室不同的工作环境。

具体到这次实践中的收获,也有一些观念上的转变。之前我一直认为仿制药就是低端的代名词,其中包含的技术含量低,效益也小。而实际上不论何种研发模式的制药企业,能在市场竞争中存活下来,一定的创新能力都是必不可少的。北京福元药业为代表的仿制药企业,其工作内容是针对外国专利到期药品的仿制并在国内上市,创新之处在于竞争之中如何尽早完成仿制创造收益;江苏先声药业则是以自主研发为主的药企,创新点在于新药物的发现;上海合全药业是专注于研发外包的CDMO公司,根据国内外客户的订单进行药物研发工作,创新之处则在于在客户要求的时间限制内高质量完成研发任务。以上三家公司所代表的三类制药企业,他们都在各自的领域进行着创新。传新内容因研发模式而异,但共同点在于,这些企业的创新都集中在为企业创造利润的部分,也就是说这里体现出了企业的逐利本质,而创新可以扩大企业的收益,缺少创新则有可能在大的竞争环境中遭到淘汰。

这次实践中我还发现几家药企与高校的合作都很少,这背后体现了高效和企业里研究动机的区别,一是为了兴趣,一是为了利益。相比于校企合作,企业与企业之间的合作更加紧密。先声与合全药业都有丰富的企业间合作,合作模式有百家汇这样类似孵化器的平台,也有研发外包这种产业链分工趋向于专门化的结果。这样的合作可以推动更多有创新想法的创业者有实现想法的机会,也能让企业在某个产业链上的某个环节将创新能力发挥到极致。企业之间没有为了所谓“商业机密”而闭门造车,这种积极的合作关系让我感觉到这个行业在国内的发展前景还是很广阔的,良性的合作为创新的种子提供了优质的土壤。

除了关于制药企业和这个产业的相关情况,本次调研对于我们自身的发展规划也有很大指导意义。企业里的科研氛围与学校的科研氛围大不相同,但两者都是在做着实际的贡献:学校的科研为了突破人类知识的界限,而企业的科研在于为社会创造价值。作为化学系的学生,未来的职业发展想必会与科研挂钩,而我们也就需要尽早确认未来的发展方向,究竟以哪一种形式创造知识。正如合全药业刘主任说的,行业没有高低之分,作为清华北大这样学校的毕业生,要做的更多是回应社会的期待,用自己的知识和能力推动社会的发展和进步。这也的确是我们作为新时代的青年,应该去考虑的。

